\documentclass{minimal}
\usepackage{xltxtra}
\usepackage{fontspec}
\usepackage{polyglossia}
\setmainlanguage{russian}
\setotherlanguage{english}
\setmainfont{Times New Roman}
\newfontfamily{\cyrillicfont}{Times New Roman}
\usepackage{amsmath}
\usepackage{tikz}
\usetikzlibrary{arrows.meta}
\begin{document}
    Давай поразмышляем, основное соотношение связывающее параметры газа, это:
    $$\frac{pV}{T}=const$$

    1. Изохорный процесс 1-2. Давление $p$ растет, объем $V$ не меняется, температура $T$ растёт

    На графике $pV$ изображается вертикальной линией идущей снизу вверх - \textbf{верно}

    На графике $pT$ изображается наклонной прямой линией, идущей снизу вверх, слева на право - \textbf{верно}

    На графике $VT$ изображается горизонтальной линией идущей слева на право - \textbf{верно}

    2. Изотермический процесс 2-3. Давление $p$ падает, объём $V$ растет, температура не меняется.

    На графике $pV$ изображается гиперболой идущей сверху вниз и слева на право - \textbf{верно}

    На графике $pT$ изображается вертикальной линией, идущей сверху вниз - \textbf{верно}

    На графике $VT$ изображается вертикальной линией идущей снизу вверх - \textbf{верно}

    3. Изобарный процесс 3-1. Давление $p$ не меняется, объем $V$ уменьшается, температура $T$ падает.

    На графике $pV$ изображается горизонтальной линией идущей справа налево - \textbf{верно}

    На графике $pT$ изображается горизонтальной линией, идущей справо налево - \textbf{верно}

    На графике $VT$ изображается наклонной линией идущей сверху вниз и справо налево - \textbf{верно}

\end{document}