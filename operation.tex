\documentclass{article}
\usepackage{xltxtra}
\usepackage{fontspec}
\usepackage{polyglossia}
\setmainlanguage{russian}
\setotherlanguage{english}
\setmainfont{Times New Roman}
\newfontfamily{\cyrillicfont}{Times New Roman}
\usepackage{amsmath}
\begin{document}
    Как делать сложение, вычитание, умножение и деление в различных системх отсчета. При работе с компьютерами наиболее часто используются системы исчисления по основанию 2, 8, 16. Сложения вычитание, умножение и деление в них осуществляется по правилам аналогичным 10 системе, только перенос в старший разряд или заем из страшего разряда осуществляется не по 10, а по основанию системы исчисления. Так например восьмеричной системе $5_8 + 5_8 = 12_8$ так как уже при $8_{10}$ производится перенос в старший рзряд. $5_{10} + 3_{10} = 8_{10}$, что соответствует $10_8$, кроме того еще остается $2_8$, что в конечном дает $12_8$

    Произведем все 4 действия над $15_{10}$ и $3_{10}$

    Сложение:

    Двоичная:
    $15_{10} = 1111_2$ и $3_{10} = 11_2$
    $$
    \begin{array}{r}
        +
        \begin{array}{r}
            1111\\
            11\\
        \end{array} \\
        \begin{array}{r}
            \hline
            10010
        \end{array}
    \end{array}
    $$

    Восьмиричная:
    $15_{10} = 17_8$ и $3_{10} = 3_8$ 
    $$
    \begin{array}{r}
        +
        \begin{array}{r}
            17 \\
            3 \\
        \end{array} \\
        \begin{array}{r}
            \hline
            22
        \end{array}
    \end{array}
    $$

    Шестнадцатиричная:
    $15_{10} = F_{16}$ и $3_{10} = 3_{16}$
    $$
    \begin{array}{r}
        +
        \begin{array}{r}
        F \\
        3 \\
        \end{array} \\
        \begin{array}{r}
        \hline
        12
        \end{array}
    \end{array}
    $$

    Вычитание:

    Двоичная:
    $$
    \begin{array}{r}
        -
        \begin{array}{r}
            1111 \\
            11 \\
        \end{array} \\
        \begin{array}{r}
            \hline
            1100 \\
        \end{array}
    \end{array}
    $$

    Восьмиричная:
    $$
     \begin{array}{r}
        -
        \begin{array}{r}
            17 \\
            3 \\
        \end{array} \\
        \begin{array}{r}
            \hline
            14
        \end{array}
    \end{array}
    $$

    Шестнадцатиричная:
    $$
    \begin{array}{r}
        -
        \begin{array}{r}
            F \\
            3 \\
        \end{array} \\
        \begin{array}{r}
            \hline
            C
        \end{array}
    \end{array}
    $$

    Умножение:

    Двоичная:
    $$
    \begin{array}{r}
        \times
        \begin{array}{r}
            1111 \\
            11 \\
            \hline
        \end{array} \\
        \begin{array}{r}
            1111 \\
            1111\phantom{0} \\
            \hline
            101101
        \end{array}
    \end{array}
    $$

    Восьмиричная:
    $$
    \begin{array}{r}
        \times
        \begin{array}{r}
            17 \\
            3 \\
        \end{array} \\
        \begin{array}{r}
            \hline
            55
        \end{array}
    \end{array}
    $$

    Шестнадцатиричная:
    $$
    \begin{array}{r}
        \times
        \begin{array}{r}
            F \\
            3 \\
        \end{array} \\
        \begin{array}{r}
            \hline
            2D
        \end{array}
    \end{array}
    $$

    Деление:

    Двоичная:
    $$
    \arraycolsep=0.01em
    \begin{array}{rrrr@{\,}r|l}
        1&1&1&1&&\,11\\
        \cline{6-6}
        1&1&&&&\,101\\
        \cline{1-2}
        &&1&1\\
        &&1&1\\
        \cline{3-4}
        &&&0\\
    \end{array}
    $$

    Восьмиричная:
    $$
    \arraycolsep=0.01em
    \begin{array}{rr@{\,}r|l}
        1&7&&\,3\\
        \cline{4-4}
        1&7&&\,5\\
        \cline{1-2}
        &0\\
    \end{array}
    $$

    Шестнадтиричная:
    $$
    \arraycolsep=0.01em
    \begin{array}{r@{\,}r|l}
        F&&\,3\\
        \cline{3-3}
        F&&\,5\\
        \cline{1-1}
        0\\
    \end{array}
    $$




\end{document}