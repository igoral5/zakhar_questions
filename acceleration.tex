\documentclass{minimal}
\usepackage{xltxtra}
\usepackage{fontspec}
\usepackage{polyglossia}
\setmainlanguage{russian}
\setotherlanguage{english}
\setmainfont{Times New Roman}
\newfontfamily{\cyrillicfont}{Times New Roman}
\usepackage{amsmath}
\usepackage{tikz}
\usetikzlibrary{arrows.meta}
\begin{document}
	Задача № 1

	Ускорение определяется по формуле:
	\[
		a_x=\frac{v_x - v_{x0}}{\Delta t}
	\]
	Если смотреть на график проекции скорости $v_x$ для времени $t=4.5~\text{с}$ имеем 

	$v_{x0}=12~\frac{\text{м}}{\text{c}}$, 

	$v_x=0~\frac{\text{м}}{\text{c}}$,

	$\Delta t=1~\text{с}$, 

	откуда получаем:
	\[
		a_x = \frac{0~\frac{\text{м}}{\text{c}} - 12~\frac{\text{м}}{\text{c}}}{1~\text{с}}=-12~\frac{\text{м}}{\text{с}^2}
	\]

	Задача № 2


	Закон Гука
	\[
		F_{\text{упр}}=-k \cdot \Delta x
	\]

	где $k$ - коэффициент упругости,

	$\Delta x$ - величина деформации.

	Отсюда:
	\[
		k=\frac{F_{\text{упр}}}{\Delta x}=\frac{12.5~\text{Н}}{0.05~\text{м}}=250~\frac{\text{Н}}{\text{м}}
	\]

\end{document}